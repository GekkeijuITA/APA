\documentclass[12pt]{article}
\usepackage[italian]{babel}
\usepackage{geometry}
\usepackage{amsmath}
\usepackage{graphicx}
\usepackage{amssymb}
\usepackage{tikz}
\usepackage{array}

\geometry{margin=2cm}
\usetikzlibrary{shapes.geometric, arrows}
\tikzstyle{startstop} = [rectangle, rounded corners, minimum width=3cm, minimum height=1cm,text centered, draw=black, fill=red!30]
\tikzstyle{io} = [trapezium, trapezium left angle=70, trapezium right angle=110, minimum width=3cm, minimum height=1cm, text centered, draw=black, fill=blue!30]
\tikzstyle{process} = [rectangle, minimum width=3cm, minimum height=1cm, text centered, draw=black, fill=orange!30]
\tikzstyle{decision} = [diamond, minimum width=3cm, minimum height=1cm, text centered, draw=black, fill=green!30]
\tikzstyle{arrow} = [thick,->,>=stealth]

\title{Capitolo 1.6}
\author{Lorenzo Vaccarecci}
\date{5 Marzo 2024}

\graphicspath{{./Immagini/}}

\begin{document}
\maketitle
\section{Esempi di progettazione e analisi di algoritmi iterativi}
\subsection{Esempio didattico}
// Pre: $x=x_{0} \wedge  y=y_{0} \text{ con } x_{0},y_{0} \in \mathbb{N}$
\begin{verbatim}
    while(x!=0) {
        x = x-1;
        y = y+1;
    }
\end{verbatim}
// Post: $y=x_{0}+y_{0}$ (viene eseguito $x_{0}$ volte)\\
Se vale Pre all'inizio deve valere Post alla fine. (condizioni sullo stato)\\
Come si pone? Trovo $INV$ (invariante)  t.c.
\begin{enumerate}
    \item vale all'inizio $pre \Rightarrow INV$
    \item alla fine garantire la Post $INV \wedge  \lnot B \Rightarrow Post$
    \item // INV $\wedge$ B \\ C \\ // INV
    \item Esiste una funzione di terminazione cioè una possibilità (dipende dallo stato) tale che sugli stati per cui vale INV $\wedge $ B (cioè se entro nel ciclo) \begin{itemize}
        \item descresce eseguendo $C$ strettamente
        \item è limitata inferiormente
    \end{itemize}
\end{enumerate}
\textit{INV approssimazione di Post}\newpage
Proviamo che 1+2+3 valgono.
\begin{center}
    \begin{tikzpicture}[node distance=2cm]
        \node (start) [startstop] {Pre};
        \node (dec1) [decision, below of=start] {B};
        \node (pro1) [process, below left of=dec1, xshift=-0.5cm, yshift=-0.5cm] {C};
        \node (end) [startstop, below right of=dec1, xshift=0.5cm, yshift=-0.5cm] {Post};
        \draw [arrow] (start) -- (dec1) node[midway, right] {INV};
        \draw [arrow] (dec1) -- node[anchor=east] {INV B} (pro1);
        \draw [arrow] (dec1) -- node[anchor=west] {INV $\lnot$B} (end);
        \draw [arrow] (pro1) |- (start);
    \end{tikzpicture}
\end{center}
\begin{center}
    1+2+3 $\Rightarrow$
    \begin{tabular}{| m{20em} | m{5em} |}
        se  vale Pre prima di while(B)C e il ciclo termina, poi vale Post & correttezza parziale\\
        \hline
        se vale Pre prima di while(B)C allora il ciclo termina e vale Post & correttezza totale
    \end{tabular}
\end{center}
+4 = quantità dipendente dallo stato che descresce strettamente a ogni iterazione $\rightarrow$ permette di determinare il tipo di correttezza
\end{document}