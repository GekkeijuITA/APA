\documentclass[12pt]{article}
\usepackage[italian]{babel}
\usepackage{geometry}
\usepackage{amsmath}
\usepackage{graphicx}
\usepackage{amssymb}
\usepackage{tikz}

\geometry{margin=2cm}

\title{Capitolo 2.4-2.5-2.6ish}
\author{Lorenzo Vaccarecci}
\date{25 Marzo 2024}

\graphicspath{{./Immagini/}}

\begin{document}
\maketitle
\section{Algoritmo di Prim}
\begin{description}
    \item[\textbf{Minimo albero ricoprente}] Dato un grafo $G$ connesso, non orientato e pesato, un minimo albero ricoprente di $G$ è un albero ricoprente di $G$ in cui la somma dei pesi degli archi è minima.
\end{description}
Un minimo albero ricoprente di $G$ è quindi un sottografo di $G$ tale che:
\begin{itemize}
    \item sia un albero libero, ossia connesso e aciclico
    \item contenga tutti i nodi di $G$
    \item la somma dei pesi degli archi sia minima
\end{itemize}
Simile a Dijkstra, ma si prende ogni volta, fra tutti i nodi adiacenti a quelli per cui si è già trovato il minimo (neri), quello connesso a un nodo nero dall'arco di costo minimo, cioè si cerca il nodo "più vicino" all'albero già costruito (poi, come in Dijkstra, si aggiornano gli altri nodi).
\begin{verbatim}
    Prim(G,s)
        for each (u nodo in G) marca u come non visitato
        for each (u nodo in G) dist[u] = inf
        parent[s] = null; dist[s] = 0
        Q = heap vuoto
        for each (u nodo in G) Q.add(u, dist[u])
        while(Q non vuota)
            u = Q.getMin()
            marca u come visitato (nero)
            for each ((u,v) arco in G)
                if(v non visitato && c_{u,v} < dist[v])
                    parent[v] = u; dist[v] = c_{u,v}
                    Q.changePriority(v, dist[v])
\end{verbatim}
A differenza di quanto accade in Dijkstra, occorre un controllo esplicito che i nodi adiacenti al nodo $u$ estratto dalla coda non siano già stati visitati, perchè non è detto che per un nodo $v$ già visitato il test $c_{u,v} < $ dist[v] sia falso.
\subsection{Correttezza}
Sia $T$ l'albero di nodi neri (non in Q) corrente. L'invariante è composta di due parti:
\begin{enumerate}
    \item $T \subseteq MST$ per qualche $MST$ minimo albero ricoprente di $G$
    \item per ogni nodo $u\neq s$ in Q dist[u] = costo minimo di un arco che collegaa $u$ a un nodo nero
\end{enumerate}
L'invariante vale all'inizio:
\begin{enumerate}
    \item vale banalmente perchè $T$ è vuoto (non ci sono nodi neri)
    \item vale banalmente perchè non ci sono nodi neri e la distanza è  per tutti infinito, tranne che per s
\end{enumerate}
L'invariante si mantiene:
\begin{enumerate}
    \item Viene estratto dalla coda un nodo $u$ con dist[u] minima, cioè connesso a un nodo nero $y$ da un arco di costo minimo tra tutti quelli che "attraversano la frontiera", cioè uniscono un nodo nero a un nodo non nero. L'arco $(y,u)$ diventa quindi parte dell'albero di nodi neri corrente.
    \item L'insieme dei nodi risulta modificato per l'aggiunta di $u$, quindi occorre ripristinare l'invarinate (2), controllando se per qualche nodo $v$ non nero e adiacente a $u$ l'arco $(u,v)$ ha costo minore del precedente arco che univa $v$ a un nodo nero, e in tal caso aggiornare l'arco e la distanza.
\end{enumerate}
Postcondizione: all'uscita dal ciclo tutti i nodi sono neri, quindi $T$ connette tutti i nodi, cioè è un albero ricoprente di $G$. Per l'invariante (1), $T\subseteq MST$ per qualche $MST$ minimo albero ricoprente, quindi $T=MST$.\\
L'analisi della complessità è esattamente la stessa dell'algoritmo di Dijkstra, quindi $O((n+m)\log n)$.
\section{Algoritmo di Kruskal}
\section{Ordinamento topologico}
\end{document}