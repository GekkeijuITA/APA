\documentclass[12pt]{article}
\usepackage[italian]{babel}
\usepackage{geometry}
\usepackage{amsmath}
\usepackage{graphicx}
\usepackage{amssymb}

\geometry{margin=2cm}

\title{Capitolo 1.3}
\author{Lorenzo Vaccarecci}
\date{27 Febbraio 2024}

\graphicspath{{./Immagini/}}
\newtheorem{definition}{Definizione}
\newtheorem{proposition}{Proposizione}

\begin{document}
\maketitle
\section{Correttezza di algoritmi ricorsivi}
\begin{definition}
    \textbf{Definizione induttiva} Una definizione induttiva R è un insieme di regole \(\frac{Pr}{c}\) dove Pr è un insieme detto delle premesse e c è un elemento, detto la conseguenza della regola. L'insieme definito induttivamente \(\mathcal{I}\) è il più piccolo tra gli insiemi X chiusi rispetto a R, ossia tali che, per ogni regola \(\frac{Pr}{c} \in R, \text{ se } Pr \subseteq X \text{ allora } c \in X\). 
\end{definition}
In genere una definizione induttiva è data attraverso una qualche descrizione finita "a parole".\\
\\\textbf{Principio di induzione (forma generale)} Sia \(R\) una definizione induttiva, \(\mathcal{I}\) l'insieme da essa definito, e \(P\) un predicato su \(U\) con \(\mathcal{I} \subseteq U\). Se\\
\textbf{Base} \(P(c)\) vale per ogni \(\frac{\emptyset}{c} \in R\)\\
\textbf{Passo induttivo} (\(P(d)\) vale per ogni \(d \in Pr\)) implica che valga \(P(c)\) per ogni \(\frac{Pr}{c} \in R, Pr \neq \emptyset\) allora \(P(d)\) vale per ogni \(d \in \mathcal{I}\)
\end{document}