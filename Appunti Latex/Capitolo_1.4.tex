\documentclass[12pt]{article}
\usepackage[italian]{babel}
\usepackage{geometry}
\usepackage{amsmath}
\usepackage{graphicx}
\usepackage{amssymb}
\usepackage{qtree}
\usepackage{tikz}

\geometry{margin=2cm}

\title{Capitolo 1.4 - 1.5}
\author{Lorenzo Vaccarecci}
\date{4 Marzo 2024}

\graphicspath{{./Immagini/}}

\begin{document}
\maketitle
\section{Correttezza di algoritmi ricorsivi}
\subsection{Torre di Hanoi}
\textit{Devo spostare tutti i dischi da A a C, senza mai mettere un disco più grande su uno più piccolo.}
\begin{center}
    \begin{tikzpicture}
        \draw (0,0) -- (0,2);
        \draw[draw=blue] (-1,0) rectangle ++(2,0.3);
        \draw[draw=green] (-0.75,0.3) rectangle ++(1.5,0.3);
        \draw[draw=red] (-0.5,0.6) rectangle ++(1,0.3);
        \draw[draw=orange] (-0.25,0.9) rectangle ++(0.5,0.3);
        \draw (2,0) -- (2,2);
        \draw (3,0) -- (3,2);
        \node at (0,-0.25) {A};
        \node at (2,-0.25) {B};
        \node at (3,-0.25) {C};
    \end{tikzpicture}
\end{center}
Pensiamo in modo ricorsivo:
\begin{itemize}
    \item $n=1 \rightarrow$ da $A$ a $C$
    \item $n+1 \rightarrow$ sposto $1\dots n$ da $A$ a $B$, sposto $n+1$ da $A$ a $C$, sposto $1\dots n$ da $B$ a $C$
\end{itemize}
\begin{verbatim}
    Hanoi(n, from, aux, to) {
        if(n==1) move(from, to);
        else {
            Hanoi(n-1, from, to, aux); // T(n-1)
            move(from, to); // 1
            Hanoi(n-1, aux, from, to); // T(n-1)
        }
    }
\end{verbatim}
\textbf{Complessità:} $T(n)$= numero di mosse(move) per $n$ dischi. $\rightarrow$ Lo possiamo esprimere ricorsivamente (induttivamente):
\begin{description}
    \item[Relazione di ricorrenza]
    \item[]  $T(1)=1$
    \item[] $T(n>1)=T(n-1)+1+T(n-1)$
\end{description}
Una tecnica che spesso consente di individuare la soluzione di una relazione di ricorrenza consiste nell'espandere successivamente le chiamate ricorsive nella definizione di $T(n)$, chiamata anche \textit{metodo "empirico"}.\newpage
Un modo alternativo di ottenere lo stesso risultato è espandendo "per livelli" (dell'albero di ricorsione), nel modo illustrato sotto:
\begin{center}
    \Tree
    [.Hanoi(n) 
        [.Hanoi(n-1) 
            [.Hanoi(n-2)
                {\ldots}
            ]
            [.Hanoi(n-2) ]
        ]
        [.Hanoi(n-1)
            [.Hanoi(n-2) ]
            [.Hanoi(n-2)
                {\ldots}
            ]
        ] 
    ]
\end{center}
Una volta individuata la soluzione in uno dei due modi, si può verificare rigorosamente la correttezza per induzione aritmetica.\\
Si ha quindi che $T(n)=\Theta(2^{n})$, mentre la complessità in spazio corrisponde all'altezza dell'albero di ricorsione, ossia alla massima profondità dello stack, quindi è $S(n)=\Omega (n)$.\\
\textbf{Induzione aritmetica}:
\begin{itemize}
    \item n=1: 2-1=1
    \item n$>$1: \begin{itemize}
        \item Ipotesi induttiva: mosse richieste per n dischi $2^{n}-1$
        \item mosse per $n+1$ dischi: $2^{n+1}-1$
    \end{itemize}
\end{itemize}
Complessità del problema $O(2^{n})$ e $\Omega (2^{n}) \rightarrow$ algoritmo esponenziale  quindi problema intrattabile (e chiuso). Esiste un algoritmo migliore? No.\\
Prova: per induzione aritmetica, per spostare $n$ dischi ci vogliono almeno $2^{n}-1$ mosse. Vero per $n=1$\\
Ipotesi induttiva: per spostare $n$ dischi ci vogliono $2^{n}-1$ mosse.\\
Tesi: per spostare $n+1$ dischi ci vogliono $2^{n+1}-1$ mosse.\\ 
\end{document}