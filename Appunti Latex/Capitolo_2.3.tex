\documentclass[12pt]{article}
\usepackage[italian]{babel}
\usepackage{geometry}
\usepackage{amsmath}
\usepackage{graphicx}
\usepackage{amssymb}
\usepackage{tikz}

\geometry{margin=2cm}

\title{Capitolo 2.3}
\author{Lorenzo Vaccarecci}
\date{18 Marzo 2024}

\graphicspath{{./Immagini/}}

\begin{document}
\maketitle
\section{Microripasso Heap}
\begin{itemize}
    \setlength\itemsep{0em}
    \item heap = albero binario
    \item ogni nodo ha priorità minore dei figli (minimo=radice)
    \item quasi completo a sinistra
    \item bilanciato (altezza $\log n$)
\end{itemize}
\section{Algoritmo di Dijsktra}
\begin{description}
    \item[\textbf{Grafo pesato}] Un grafo pesato è un grafo $G=(V,E)$ dove a ogni arco è associato un peso o costo attraverso una funzione $c_{\_}:E\rightarrow\mathbb{R}$. Il costo di un cammino da $s$ a $t$ è la somma dei pesi o costi degli archi che compongono il cammino. Un cammino da $s$ a $t$ si dice \textit{minimo} se ha peso minimo (detto distanza) fra tutti i cammini da $s$ a $t$.  
\end{description}
Problema: dato un grafo orientato pesato $G$, con pesi non negativi e dati un nodo di partenza $s$ e un nodo di arrivo $t$, trovare un cammino minimo da $s$ a $t$. Generalizzazione: dati un grafo orientato pesato $G$ con pesi non negativi e un nodo di partenza $s$, trovare per ogni nodo $u$ del grafo il cammino minimo da $s$ a $u$.\\
Algoritmo di Dijkstra, idea: è una visita in ampiezza in cui a ogni passo:
\begin{itemize}
    \setlength\itemsep{0em}
    \item per i nodi già visitati si ha la distanza e l'albero $T$ di cammini minimi
    \item per ogni nodo non ancora visitato si ha distanza provvisoria = lunghezza minima di un cammino in $T$ più un arco
    \item si estrae dalla coda un nodo a distanza provvisoria minima (che risulta essere la distanza)
    \item si aggiornano le distanze provvisorie dei nodi adiacenti al nodo estratto tenendo conto del nuovo arco in $T$
\end{itemize}\newpage
\subsection{Pseudocodice}
\begin{verbatim}
    Dijkstra(G,s)
        for each (u nodo in G) dist[u] = inf
        parent[s] = null
        dist[s] = 0
        Q = heap vuoto
        for each (u nodo in G) Q.add(u,dist[u])
        while (Q non vuota)
            u = Q.getMin()
            for each((u,v) arco in G)
                if(dist[u] + c_{uv} < dist[v])
                    parent[v] = u
                    dist[v] = dist[u] + c_{uv}
                    Q.changePriority(v,dist[v])
\end{verbatim}
\subsection{Correttezza}
Sia $d(u)$ la distanza (lunghezza di un cammino minimo) da $s$ a $u$. L'invariante è composta di due parti:
\begin{enumerate}
    \item $\forall u\notin H \quad dist[u]=d(u)$
    \item per ogni nodo $u$ in $Q$ $dist[u]=d_{\backslash Q}(u)$= lunghezza di un cammino minimo da $s$ a $u$ i cui nodi, tranne $u$, non sono in $Q$, ossia sono nodi "neri".
\end{enumerate}
L'invariante vale all'inizio:
\begin{enumerate}
    \item vale banalmente perchè non ci sono nodi neri ($H$ vuoto)
    \item vale banalmente perchè la distanza è per tutti infinito, tranne che per $s$ per cui vale 0 (non ci sono cammini che usano solo nodi neri) 
\end{enumerate}
Postcondizione: al termine dell'esecuzione dell'algoritmo la coda è vuota, quindi tutti i nodi sono neri. Poichè vale l'invariante (1), per ogni nodo è stato trovato il cammino minimo da $s$.
\end{document}