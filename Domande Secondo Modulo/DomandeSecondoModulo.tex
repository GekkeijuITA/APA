\documentclass[12pt]{article}

\usepackage[italian]{babel}
\usepackage{amsmath}
\usepackage{amssymb}
\usepackage{geometry}
\usepackage{xcolor}
\usepackage{listings}

\geometry{margin=2cm}

\newcounter{questioncounter}
\newcommand{\question}[1]{
    \stepcounter{questioncounter}
    \textbf{\\\textcolor{red}{\arabic{questioncounter}. #1}}\\
}

\let\olditemize\itemize
\renewcommand\itemize{\olditemize\setlength\itemsep{0em}}

\definecolor{codegreen}{rgb}{0,0.6,0}
\definecolor{codegray}{rgb}{0.5,0.5,0.5}
\definecolor{codepurple}{rgb}{0.58,0,0.82}
\definecolor{backcolour}{rgb}{0.95,0.95,0.92}

\lstdefinestyle{mystyle}{
    backgroundcolor=\color{backcolour},   
    commentstyle=\color{codegreen},
    keywordstyle=\color{magenta},
    numberstyle=\tiny\color{codegray},
    stringstyle=\color{codepurple},
    basicstyle=\ttfamily\footnotesize,
    breakatwhitespace=false,         
    breaklines=true,                 
    captionpos=b,                    
    keepspaces=true,                 
    numbers=left,                    
    numbersep=5pt,                  
    showspaces=false,                
    showstringspaces=false,
    showtabs=false,                  
    tabsize=2
}

\lstset{style=mystyle}


\title{Domande Orale Secondo Modulo: Analisi e Progettazione di Algoritmi (APA)}
\author{}
\date{}

\begin{document}
\maketitle
\section{Ordinamento}
\question{Costruisci l’input peggiore per una versione deterministica di QuickSort che sceglie come
pivot (a) il primo elemento di una sequenza e (b) l’elemento mediano}
\question{Supponi di avere una probabilità congiunta di 100 variabili casuali di Bernoulli non indipendenti. Se devi calcolare il valore atteso della somma partendo da questa distribuzione
quanti sono i termini da sommare? Quanti invece se utilizzi il fatto che il valore atteso di una somma è uguale alla somma dei valori attesi? Che cosa cambia se fossero
indipendenti?}
\question{Nell’assunzione che la sequenza S consista dei primi n numeri naturali, spiega per quale
motivo la probabilità che i sia confrontato con j può essere espressa come}
\begin{equation*}
    \frac{2}{\lvert i - j \rvert + 1}
\end{equation*}
\section{Teoria dei giochi}
\question{Determina il migliore degli scenari possibili per Roberta e Carlo data una matrice M dei
pagamenti in un gioco a somma zero}
\question{Che cosa sono le strategie miste e, in particolari, le strategie miste ottimali?}
\question{Che cosa puoi fare per ottenere un limite inferiore al costo computazionale atteso di un
algoritmo randomizzato?}
\section{Valutazione dell'albero di un gioco}
\question{Produci un esempio di un albero di un gioco per il quale un algoritmo deterministico deve
valutare tutte le foglie}
\question{Dimostra che nel caso base di un albero binario di un gioco un algoritmo randomizzato
controlla, in media, 3 delle 4 foglie}
\section{Taglio minimo}
\question{Spiega il motivo per il quale se k è la cardinalità di un taglio minimo per un grafo G con n vertici, il numero di archi non può essere minore di nk/2.}
\question{Sia $G$ un grafo connesso di $n$ vertici e taglio minimo uguale a $k$. Se $E_{i}$ è l’evento di non selezionare un arco di $S$ all’$i$-esima iterazione di $MCMinCut$, spiega il motivo per cui}
\begin{equation*}
    Pr(E_{i}|E_{1}, E_{2}, \ldots, E_{i-1}) \geq \frac{2}{n-i+1}
\end{equation*}
\question{Se un algoritmo Monte Carlo ottiene il risultato corretto con probabilità p, quante volte
deve essere ripetuto per restituire il risultato corretto almeno una volta con probabilità del
99.9\%?}
\section{Accordo bizantino}
\question{Per quali motivi raggiungere il consenso distribuito è un problema complicato?}
\question{Quali sono le specifiche per il raggiungimento del consenso bizantino e quali le motivazioni per i vincoli di consenso e validità?}
\question{Spiega il motivo per cui se $n = 3f$ e $T = 2f$ il protocollo $MGByzantineGeneral$ potrebbe non raggiungere mai un accordo}
\question{Siano dati $n = 3f +1$ processi, $f$ dei quali inaffidabili. Per quale motivo, se $n$ è abbastanza grande e il bit di ogni processo affidabile è inizializzato uniformemente a caso, i $2f +1$ processi affidabili raggiungono l’accordo dopo il primo round con grande probabilità?}
\section{Test di primalità}
\question{Verifica che gli elementi di $Z^{+}_{10}$ che non sono coprimi di 10 non ammettono inverso e quindi non possono appartenere a $Z^{*}_{10}$}
\question{Determina gli elementi di $Z^{*}_{12}$ e verifica che ogni elemento è l’inverso di se stesso.}
\question{Eseguendo a mano i passi di MCPrimalityTest verifica che per ogni $a \in \{2, \ldots , 5\}$, 7 è probabilmente primo.}
\section{Verifiche di uguaglianza}
\question{Per quale motivo verificare il prodotto di due matrici $n \times n$ richiede solo $O(n^{2})$ moltiplicazioni?}
\question{Come costruiresti un insieme di numeri primi minori di $l^{2}$?}
\question{Poni $l = 8$ e costruisci due file a e b di 8 bit con $a \neq b$. Campiona un numero primo a caso tra 2 e 64 e confronta le due fingerprint ottenute. Spiega quale strategia consentirebbe di determinare tutti i numeri primi per i quali le fingerprint sono uguali.}
\end{document}