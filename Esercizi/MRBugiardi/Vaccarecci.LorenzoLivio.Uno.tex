\documentclass[12pt]{article}

\usepackage[italian]{babel}
\usepackage{amsmath}
\usepackage{verse}
\usepackage{geometry}
\usepackage{graphicx}
\usepackage{amssymb}
\usepackage{listings}
\usepackage{xcolor}

\geometry{margin=2cm}

\let\olditemize\itemize
\renewcommand\itemize{\olditemize\setlength\itemsep{0em}}

\definecolor{codegreen}{rgb}{0,0.6,0}
\definecolor{codegray}{rgb}{0.5,0.5,0.5}
\definecolor{codepurple}{rgb}{0.58,0,0.82}
\definecolor{backcolour}{rgb}{0.95,0.95,0.92}

\lstdefinestyle{mystyle}{
    backgroundcolor=\color{backcolour},   
    commentstyle=\color{codegreen},
    keywordstyle=\color{magenta},
    numberstyle=\tiny\color{codegray},
    stringstyle=\color{codepurple},
    basicstyle=\ttfamily\footnotesize,
    breakatwhitespace=false,         
    breaklines=true,                 
    captionpos=b,                    
    keepspaces=true,                 
    numbers=left,                    
    numbersep=5pt,                  
    showspaces=false,                
    showstringspaces=false,
    showtabs=false,                  
    tabsize=2
}

\lstset{style=mystyle}

\title{MR Bugiardi}
\author{Lorenzo Livio Vaccarecci (matr. 5462843)}
\date{A.A. 2023/2024}

\begin{document}
\maketitle
\section{Codice}
\lstinputlisting[language=Python]{PrimalityTest.py}
Sappiamo che per i numeri di Carmichael i bugiardi sono i coprimi e per sicurezza controllo se Hn è uguale ai coprimi.
\begin{itemize}
    \item I bugiardi per \textbf{561} sono tutti i numeri da 1 a 560 tranne tutti i multipli di [3,11,17] (compresi)
    \item I bugiardi per \textbf{1105} sono tutti i numeri da 1 a 1104 tranne tutti i multipli di [5,13,17] (compresi)
    \item I bugiardi per \textbf{1729} sono tutti i numeri da 1 a 1728 tranne tutti i multipli di [7,13,19] (compresi)
    \item I bugiardi per \textbf{2465} sono tutti i numeri da 1 a 2464 tranne tutti i multipli di [5,17,29] (compresi)
    \item I bugiardi per \textbf{2821} sono tutti i numeri da 1 a 2820 tranne tutti i multipli di [7,13,31] (compresi)
    \item I bugiardi per \textbf{6601} sono tutti i numeri da 1 a 6600 tranne tutti i multipli di [7,23,41] (compresi)
    \item I bugiardi per \textbf{8911} sono tutti i numeri da 1 a 8910 tranne tutti i multipli di [7,19,67] (compresi)
\end{itemize}


\end{document}