\documentclass[12pt]{article}

\usepackage[italian]{babel}
\usepackage{amsmath}
\usepackage{verse}
\usepackage{geometry}
\usepackage{graphicx}
\usepackage{amssymb}
\usepackage{listings}
\usepackage{xcolor}
\usepackage{xfrac}

\geometry{margin=2cm}

\let\olditemize\itemize
\renewcommand\itemize{\olditemize\setlength\itemsep{0em}}

\title{Valore di un gioco}
\author{Lorenzo Livio Vaccarecci (matr. 5462843)}
\date{A.A. 2023/2024}

\begin{document}
\maketitle
\begin{equation*}
    M =
    \begin{pmatrix}
        4 & -1 \\
        -2 & 3
    \end{pmatrix}
\end{equation*}
\textbf{Compito 4.2}: Dimostra che $V_{R}<V_{C}$, determina le strategie miste ottimali di Roberta e Carlo e il valore del gioco.
\subsection*{a)}
\begin{equation*}
    V_{R} = \max\{-1,-2\}=-1, V_{C} = \min\{4,3\}=3
\end{equation*}
$V_{R}<V_{C}$
\subsection*{b)}
\begin{equation*}
    p = (p \quad 1-p)^{T}, q = (q \quad 1-q)^{T}
\end{equation*}
\vspace{0.01cm}
\begin{equation*}
    \begin{aligned}
        p^{T}Mq & = (p \quad 1-p) \begin{pmatrix}
            4 & -1 \\
            -2 & 3
        \end{pmatrix} \begin{pmatrix}
            q \\
            1-q
        \end{pmatrix} \\
        & = 10qp - 5q - 4p + 3 \\
        & = 10\left(qp-\frac{5}{10}q-\frac{4}{10}p+\frac{3}{10}\right) \\
        & =10\left(qp-\frac{1}{2}q-\frac{2}{5}p+\frac{2}{10}+\frac{1}{10}\right) \\
        & = 10\left(p-\frac{1}{2}\right)\left(q-\frac{2}{5}\right)+1
    \end{aligned}
\end{equation*}
\begin{equation*}
    \hat{p} = \left(\frac{1}{2}\quad\frac{1}{2}\right)^{T}
    \hat{q} = \left(\frac{2}{5}\quad\frac{3}{5}\right)^{T}
\end{equation*}
\newpage
\subsection*{c)}
Quando vengono utilizzate le strategie miste ottimali $\hat{p},\hat{q}$, il valore del gioco è:
\begin{equation*}
        V = V_{R} = V_{C} = 1
\end{equation*}
Dimostrazione:
\begin{equation*}
    \hat{p}^{T}M = \left(\frac{1}{2}\quad\frac{1}{2}\right) \begin{pmatrix}
        4 & -1 \\
        -2 & 3
    \end{pmatrix}
    = \left(1 \quad 1\right)
\end{equation*}
\begin{equation*}
    M\hat{q} = \begin{pmatrix}
        4 & -1 \\
        -2 & 3
    \end{pmatrix}\begin{pmatrix}
        \sfrac{2}{5} \\
        \sfrac{3}{5}
    \end{pmatrix} = \begin{pmatrix}
        1 \\
        1
    \end{pmatrix}
\end{equation*}
\end{document}