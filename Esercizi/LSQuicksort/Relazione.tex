\documentclass[12pt]{article}

\usepackage[italian]{babel}
\usepackage{amsmath}
\usepackage{verse}
\usepackage{geometry}
\usepackage{graphicx}
\usepackage{amssymb}
\usepackage{listings}
\usepackage{xcolor}

\geometry{margin=2cm}

\let\olditemize\itemize
\renewcommand\itemize{\olditemize\setlength\itemsep{0em}}

\definecolor{codegreen}{rgb}{0,0.6,0}
\definecolor{codegray}{rgb}{0.5,0.5,0.5}
\definecolor{codepurple}{rgb}{0.58,0,0.82}
\definecolor{backcolour}{rgb}{0.95,0.95,0.92}

\lstdefinestyle{mystyle}{
    backgroundcolor=\color{backcolour},   
    commentstyle=\color{codegreen},
    keywordstyle=\color{magenta},
    numberstyle=\tiny\color{codegray},
    stringstyle=\color{codepurple},
    basicstyle=\ttfamily\footnotesize,
    breakatwhitespace=false,         
    breaklines=true,                 
    captionpos=b,                    
    keepspaces=true,                 
    numbers=left,                    
    numbersep=5pt,                  
    showspaces=false,                
    showstringspaces=false,
    showtabs=false,                  
    tabsize=2
}

\lstset{style=mystyle}

\title{Relazione: LS Quicksort}
\author{Lorenzo Livio Vaccarecci (matr. 5462843)}
\date{A.A. 2023/2024}

\begin{document}
\maketitle
\lstinputlisting[language=Python]{LSQuicksort.py}
Dopo aver implementato l'algoritmo Quicksort Las Vegas, 
si sono utilizzate delle sequenze di numeri casuali di 
cardinalità $10^{4}$ eseguendo l'algoritmo per $R=10^{5}$ 
volte (con array generati in modo casuale) e si sono ottenuti i seguenti valori:
\begin{itemize}
    \item Valore medio($\hat{\mu}$) $\simeq $ 156533.23
    \item Deviazione standard empirica($\hat{\sigma}$) $\simeq$ 6498.49
\end{itemize}
Per calcolare i valori sono state usate le formule:
\begin{equation*}
    \hat{\mu} = \frac{1}{R}\sum_{r=1}^{R}X_{r} 
\end{equation*}
Dove $X_{r}$ è il numero di confronti effettuati al passo $r$-esimo e:
\begin{equation*}
    \hat{\sigma}^{2} = \frac{1}{R-1}\sum_{r=1}^{R}(X_{r}-\hat{\mu})^{2}
\end{equation*}
Usando il numero di confronti effettuati si può generare il seguente 
istogramma:
\begin{center}
    \includegraphics[scale=0.8]{Istogramma.png}
\end{center}
Il grafico mostra che la distribuzione dei valori è ampia (questo 
conferma la deviazione standard elevata), il picco si trova in un 
valore vicino a 150000 confermando il valore medio ottenuto.
Un'ulteriore conferma che il valore medio è quello atteso è data dalla formula:
\begin{equation*}
    \mathbb{E}[X] = \sum_{i=1}^{n-1}2\ln(n-i+1) = 2\sum_{i=1}^{n-1}\ln(n-i+1)
    = 2\sum_{i=1}^{10^{4}-1}\ln(10^{4}-i+1) \simeq 164216.47
\end{equation*}
In modo simile:
\begin{equation*}
    \mathbb{E}[X]=2n\ln(n) = 2\cdot 10^{4}\ln(10^{4}) \simeq 184206.81
\end{equation*}
Si può notare che i valori medi ottenuti dalle formule teoriche e quello
calcolato dall'implementazione dell'algoritmo sono molto simili.
Questa discrepanza può essere attribuita a diversi fattori, tra cui la natura
approssimativa delle formule teoriche e la presenza di numeri casuali. Tuttavia,
il valore medio ottenuto dal programma è ragionevole e conferma l'efficienza
dell'algoritmo.\\
Calcolata la disuguaglianza di Chebyshev usando la formula:
\begin{equation*}
    \frac{\hat{\sigma}^{2}}{(v-1)^{2}\hat{\mu}^{2}}
\end{equation*}
Abbiamo che  per il 
doppio ($2\hat{\mu}\simeq 313066.46$) e il triplo 
($3\hat{\mu} \simeq 469599.69$) del valore medio con 
$v=2 \quad v=3$ rispettivamente:
\begin{equation*}
    \frac{6498.49^{2}}{(2-1)^{2}\cdot 156533.23^{2}} \simeq 0.0017 
    \simeq 0.17\%
\end{equation*}
e
\begin{equation*}
    \frac{6498.49^{2}}{(3-1)^{2}\cdot 156533.23^{2}} \simeq 0.0004 
    \simeq 0.04\%
\end{equation*}
Calcolata anche la disuguaglianza di Markov per le stesse $v$ usando la formula:
\begin{equation*}
    \frac{\hat{\mu}}{v\hat{\mu}}
\end{equation*}
Si ottengono facilmente le due probabilità:
\begin{equation*}
    \frac{156533.23}{2\cdot 156533.23} = 0.5 = 50\%
\end{equation*}
e
\begin{equation*}
    \frac{156533.23}{3\cdot 156533.23} \simeq 0.33 \simeq 33\%
\end{equation*}
Secondo la disuguaglianza di Markov, la probabilità che il numero di confronti
sia superiore a $2\hat{\mu}$ è del 50\%, mentre la probabilità che sia
superiore a $3\hat{\mu}$ è del 33\%. Questo significa che la disuguaglianza
di Markov è molto meno cauta rispetto a quella di Chebyshev che, invece, 
fornisce un limite superiore alla probabilità che il numero di confronti
sia superiore a $2\hat{\mu}$ del 0.17\% e del 0.04\% per $3\hat{\mu}$.\\
In conclusione, i risultati ottenuti dall'implementazione dell'algoritmo 
Quicksort Las Vegas confermano la sua efficienza per la risoluzione di
problemi di ordinamenti di array molto grandi. Il valore medio e la 
deviazione standard calcolati indicano che la distribuzione dei valori
è ampia ma concentrata attorno al valore medio come confermato dall'istogramma.\\
Le disuguaglianze di Chebyshev e Markov confermano che la probabilità
che il numero di confronti sia superiore al doppio e al triplo del valore 
medio è molto bassa.
\end{document}